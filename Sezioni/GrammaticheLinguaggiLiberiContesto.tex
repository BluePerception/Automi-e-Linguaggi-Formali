\chapter{Grammatiche e linguaggi liberi dal contesto}
Facciamo subito un esempio con un linguaggio non regolare, quindi libero dal contesto.
Analizziamo il linguaggio dei palindromi con alfabeto \{0,1\} e per dimostrare che non 
è regolare usiamo il Pumping Lemma.
Consideriamo la stringa palindroma $w=0^{n}10^{n}$. Per il lemma possiamo scomporre
\textit{w} in \textit{w=xyz}, in modo tale che y consista di uno o più 0 del primo
gruppo. Di conseguenza, \textit{xz}, che dovrebbero trovarsi in $L_{pal}$ se questo 
fosse regolare, avrebbe meno 0 a sinistra dell'unico 1 rispetto a quello a destra.
Dunque \textit{xz} non può essere palindroma. 
Per definire il linguaggio $L_{pal}$ possiamo usare una definizione ricorsiva:
\begin{itemize}
\item \textsc{base}: $\varepsilon$, 0 e 1 sono in $L_{pal}$;
\item \textsc{induzione}: se w è in $L_{pal}$, allora 0w0 e 1w1 sono in $L_{pal}$.
\end{itemize}

\section{Grammatiche libere dal contesto}
Il compito della grammatica con le sue regole è uno strumento per generare tutte le
stringhe del linguaggio libero dal contesto.
Definizione di grammatica libera da contesto:
\begin{itemize}
\item[1.] Insieme finito T di simboli terminali che formano le stringhe del linguaggio
(per $L_{pal}$, 0 e 1);
\item[2.] Insieme finito V di variabili (chiamate anche \textit{non terminali}). Ogni
variabile genera un linguaggio (per $L_{pal}$ c'è solo la variabile P);
\item[3.] Un simbolo iniziale S che genera il linguaggio da definire (in $L_{pal}$
coincide con P);
\item[4.] Un insieme finito (P o R) di produzioni o regole che hanno forma 
\textbf{$testa \rightarrow corpo$};
\end{itemize}

La testa della produzione è una variabile che viene definita (parzialmente) 
dalla produzione mentre il corpo della produzione rappresenta un modo di formare 
stringhe nel linguaggio della variabile di testa ed è costituita da una stringa di
zero o più terminali e variabili. Le stringhe si formano lasciando immutati i 
terminali e sostituendo ogni variabile del corpo con una stringa appartenente al
linguaggio della variabile stessa.
\vspace{0.5cm}

Questi 4 componenti formano una grammatica libera dal contesto (CFG) che viene così
rappresentata: $G=(V,T,P,S)$, dove V è l'insieme delle variabili, T i terminali, 
P l'insieme delle produzioni ed S il simbolo iniziale).
Per quanto riguarda alla grammatica del linguaggio delle stringhe palindrome, avremo
$G_{pal}=(\{P\}, \{0,1\}, A, P)$ dove A rappresenta l'insieme delle produzioni delle
palindrome:
\begin{itemize}
\item[1.] $P \rightarrow \varepsilon$
\item[2.] $P \rightarrow 0$
\item[3.] $P \rightarrow 1$
\item[4.] $P \rightarrow 0P0$
\item[5.] $P \rightarrow 1P1$
\end{itemize}
Notazione compatta per descrivere le regole di produzione: $P \rightarrow \varepsilon
 | 0 | 1 | 0P0 | 1P1$

\section{Linguaggio di una grammatica}
Se noi abbiamo una grammatica libera da contesto, questa definirà il linguaggio 
libero da contesto da cui deriva. Abbiamo due strade per passare dalla grammatica al 
linguaggio definita dalla grammatica stessa: dalla testa al corpo 
(\textbf{derivazione}) oppure dal corpo alla testa (\textbf{inferenza ricorsiva}). 
Sono due maniere opposte per vedere la grammatica definita, ma sono equivalenti tra 
loro.

\paragraph{Derivazione}
Viene indicata con $\Rightarrow$ ed indicata la relazione tra due forme sentenziali.
Viene inoltre definita in questo modo:
\begin{defn}[Derivazione]
Sia G=(V, T, R, S) e sia $\alpha$A$\beta$ dove $\alpha$ e $\beta$ sono forme
sentenziali ed A è in V, sia $A\rightarrow\gamma$ in R, allora $\alpha$A$\beta$
\begin{itemize}
\item $\Rightarrow \alpha\gamma\beta$
\item $\Rightarrow\star$ è la chiusura riflessiva e transitiva della derivazione 
(significa zero o più passaggi)
\end{itemize}
\end{defn}

Riprendendo l'esempio di $G_{pal}$ dove $P\rightarrow\varepsilon|0|1|0P0|1P1$, 
abbiamo:
\\[0.3cm]
$P\Rightarrow 0P0\Rightarrow 00P00 \Rightarrow 001P100 \Rightarrow 0010100$
\\[0.3cm]
$P\Rightarrow ^{\ast} 0010100$

\subparagraph{Derivazione a destra e a sinistra}
Per ridurre il numero di scelte possibili nella derivazione di una stringa, spesso
è comodo imporre che a ogni passo si sostituisca la variabile all'estrema sinistra 
con il corpo di una delle sue produzioni. Tale derivazione viene detta derivazione a 
sinistra e si indica tramite $\underset{lm}{\Rightarrow}$ e 
$\overset{^{\ast}}{\underset{lm}{\Rightarrow}}$, rispettivamente per uno o molti 
passi. Analogamente vale lo stesso per la derivazione a destra, con i simboli
$\underset{rm}{\Rightarrow}$ e 
$\overset{^{\ast}}{\underset{rm}{\Rightarrow}}$.

\paragraph{Inferenza ricorsiva}
Metodo in cui prendiamo stringhe di cui conosciamo l'appartenenza al linguaggio di 
ognuna delle variabili del corpo, le concateniamo nell'ordine adeguato, con i 
terminali che compaiono nel corpo, e deduciamo che la stringa risultante sia nel
linguaggio della variabile che compare in testa.
\\[0.2cm]
Esempio di inferenza ricorsiva di 0010100:
\\[0.1cm]
Ricordando che $P\rightarrow\varepsilon|0|1|0P0|1P1$, otteniamo la seguente tabella

\begin{center}
\begin{tabular}{|c|c|c|c|c|}
\hline 
(i) & 0 & P & 2 & -- \\
\hline
(ii) & 101 & P & 5 & (i) \\
\hline 
(iii) & 01010 & P & 4 & (ii) \\
\hline
(iv) & 0010100 & P & 4 & (iii) \\
\hline
\end{tabular}
\end{center}

\paragraph{Forme sentenziali} Le derivazioni dal simbolo iniziale producono stringhe
che hanno un ruolo importante e che vengono chiamate \textit{forme sentenziali}.
Se $G=(V, T, P, S)$ è una CFG, allora ogni stringa $\alpha$ in $(V\cup T)^{\ast}$
tale che $S\stackrel{\ast}\Rightarrow \alpha$ è una forma sentenziale.





















