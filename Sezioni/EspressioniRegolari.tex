\chapter{Espressioni regolari}
Diciamo che una parola è accettata oppure no in base allo stato in cui giungiamo alla
fine di questa parola (se sono in uno stato finale, allora si tratta di una parola accettata
dall'automa altrimenti no).
Un'espressione regolare è un modo dichiarativo per descrivere un linguaggio regolare.
Abbiamo diversi tipi di operazioni sui linguaggi:

\begin{itemize}
\item \textbf{Unione}: $L\cup M=\{w:w \in L oppure w \in M\}$
\item \textbf{Concatenazione}: $L.M=\{uv : u \in L e v \in M\}$
\item \textbf{Potenze}: 
	\begin{itemize}
	\item $L^{0}=\{\varepsilon\}$
	\item $L^{1}=L$   
	\item $L^{k}=L.L...L(k volte)$
	\end{itemize}
\item \textbf{Chiusura di Kleene}: $L^{\ast} = \bigcup_{i=0}^\infty L^{i}$
\end{itemize} 

Inoltre le espressioni regolari sono costruite utilizzando 

\begin{itemize}
\item un insieme di costanti di base:
	\begin{itemize}
	\item $\varepsilon$ per la stringa vuota
	\item $\emptyset$ per il linguaggio vuoto
	\item a,b,.. per i simboli $a,b,..\in \Sigma$
	\end{itemize}
\item collegati da operatori:
	\begin{itemize}
	\item $+$ per l'unione;
	\item $\cdot$ per la concatenazione;
	\item $\ast$ per la chiusura di Kleene;
	\end{itemize}
\end{itemize}

Esistono anche delle \textbf{regole di precedenza} degli operatori:
\begin{itemize}
\item[1]Chiusura di Kleene;
\item[2]Concatenazione;
\item[3]Unione.
\end{itemize}










