\chapter{Automi stati finiti non deterministici}
E un automa che può trovarsi contemporaneamente in più stati diversi e le 
transizioni non devono per forze essere complete, per esempio: 
(immagine)
Infatti questo non può essere un DFA perchè da q0 se leggo 0 posso trovarmi
contemporaneamente in q0 e q1, in più da q1 posso muovermi solo in q2 e da q2
non posso proprio muovermi.
Un automa a stati finiti non deterministici(NFA) è una quintupla $A=(Q, \Sigma, \delta,
q0, F)$ dove
\begin{itemize}
\item Q è un insieme finito di stati;
\item $\Sigma$ è un alfabeto finito;
\item $\delta$ è una funzione di transizione che prende in input (q,a) e restituisce
un sottoinsieme di Q;
\item q0 \in Q è lo stato iniziale;
\item F \in Q è un insieme di stati finali;
\end{itemize}
Anche per i NFA abbiamo una definizione rigorosa:
\begin{itemize}
\item base: $\delta'(q, \varepsilon)={q}$
\item induzione: $\delta'(q,w)=\cup \delta(p,a)$
\end{itemize}
Data una parola, il nostro automa potrà trovarsi in uno dei tanti stadi
che siamo andati a calcolare.