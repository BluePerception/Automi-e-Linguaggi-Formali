\chapter{Automi a stati finiti deterministici}
Finire il primo slot di slide.

\subparagraph{Linguaggio accettato da un DFA}
La funzione di transizione prende in input uno stato e una parola dando in output
una nuova parola. Definizione:
\begin{itemize}
\item base: $\delta(q,\varepsilon)=q$ -> ritorna lo stadio in cui è
\item induzione: $\delta'(q,w)=\delta(\delta'(q,x),a)$ ->$\delta'$ 
rappresenta lo stato attuale e $\delta$ lo stato in cui mi troverò,
a indica l'ultima lettera della parola che voglio leggere
NB: in $\delta'$ faccio la ricorsione fino ad arrivare al caso base 
\end{itemize}
Detto ciò, possiamo definire il linguaggio accettato da A in questo modo:
$L(A)={w: \delta'(q0,w) \in F}$. Tutti i linguaggi accettati da DFA vengono
chiamati \textbf{linguaggi regolari}.