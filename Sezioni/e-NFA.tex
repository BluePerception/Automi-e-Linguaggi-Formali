\chapter{NFA con epsilon-transizioni}
Le epsilon-transizioni vengono usate per muovere l'automa di stato anche se non
viene dato nessun simbolo in input.
Un automa a stati finiti non deterministico con $\varepsilon$\textrm{-transizioni} ($
\varepsilon\textrm{-NFA}$)
è una quintupla $A=(Q, \Sigma, \delta, q_0, F)$ dove cambia solo

\begin{itemize}
\item \textbf{$\delta$} che è una funzione di transizione che prende in input uno 
stato in Q oppure un simbolo nell'alfabeto $\Sigma \cup \{\varepsilon\}$
\end{itemize}

L'eliminazione delle $\varepsilon$-transizioni procede per $\varepsilon$-chiusura 
degli stati, cioè prendendo tutti gli stati raggiungibili da q con una sequenza $
\varepsilon\varepsilon... \varepsilon$. L'$\varepsilon$-chiusura viene indicata con 
\textbf{ECLOSE(q)}.

\section{Equivalenza tra DFA e $\varepsilon$\textrm{-NFA}}
Per ogni $\varepsilon$\textrm{-NFA} E c'è una DFA D tale che L(E)=L(D), e viceversa.
Ogni stato del DFA corrisponde ad un insieme di stati chiuso per 
$\varepsilon$\textrm{-chiusura}.
Uno stato del DFA è finale se c'è \textbf{almeno} uno stato finale corrispondente 
nell'$\varepsilon$\textrm{-NFA}.


\begin{thm}
Sia $D=(Q_D, \Sigma, S_0, F_D)$ il DFA ottenuto da un $\varepsilon$\textrm{-transizioni}
E con la costruzione a sottoinsieme modificata. Allora L(D)=L(E).
\end{thm}

In altre parole, se riusciamo a dimostrare che hanno lo stesso risultato, allora i due
automi sono equivalenti.

\begin{thm}
Un linguaggio L è accettato da un DFA se e solo se è accettato da un 
$\varepsilon$\textrm{-NFA}.
\end{thm}