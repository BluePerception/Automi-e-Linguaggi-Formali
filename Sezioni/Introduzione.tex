\chapter{Introduzione}
Per iniziare, ci sono alcuni concetti di base da tenere a mente:

\begin{itemize}

\item \textbf{Alfabeto}: insieme finito e non vuoto di simboli, per esempio
$\Sigma$ = {0,1} oppure $\Sigma$ = {a,b,c,d,e,..,z};
\item \textbf{Stringa}: sequenza finita di simboli da un alfabeto $\Sigma$, per 
esempio: 011001 o abc;
\item \textbf{Stringa vuota}: stringa con zero occorrenze di simboli dell'alfabeto 
$\Sigma$, denotata da $\varepsilon$;
\item \textbf{Lunghezza di una stringa}: numero di simboli nella stringa, per 
esempio $|$w$|$ denota la lunghezza della stringa w e $|$01001$|$ = 5;
\item \textbf{Potenze di un alfabeto}: $\Sigma^{k}$ insieme delle stringhe di 
lunghezza k con simboli da $\Sigma$, per esempio preso l'alfabeto $\Sigma$={0,1}
$\Sigma^{0}$={$\varepsilon$}, $\Sigma^{1}$={0,1}, $\Sigma^{2}$={00,01,10,11}.
Viene chiamata potenza di un'alfabeto poichè può essere vista come una potenza dove 
la base è il numero di simboli dell'alfabeto e l'esponente il numero della potenza
dell'alfabeto (quindi, nell'alfabeto dei numeri binari con $\Sigma^{3}$, avremo
 $2^{3}=8$);
\item \textbf{Insieme di tutte le stringhe}: per ottenere l'insieme di tutte le 
stringhe, usiamo il simbolo \textbf{*} e scriviamo $\Sigma^{*}=\Sigma^{0} \cup 
\Sigma^{1} \cup \Sigma^{2} \cup ..$ ;
\item \textbf{Linguaggio}: dato un alfabeto $\Sigma$, chiamiamo linguaggio ogni
sottoinsieme $L\subseteq\Sigma^{*}$ (compreso anche il linguaggio vuoto che non 
contiene nessuna parola).

\end{itemize}
